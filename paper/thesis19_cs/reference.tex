%\addchapter{References}
\addchapter{参考文献}
\begin{thebibliography}{99}% 文献数が10未満の時 {9},10\UTF{FF5E}99の時 {99}

\bibitem{Material}
    Fleming, R, W., Wiebel, C., Gegenfurtner, K. (2013).
    Perceptual qualities and material classes.
    {\it Journal of Vision}, 13(8):9, 1-20.

\bibitem{Motoyoshi}
    Motoyoshi, I., Nishida, S., Sharan, L., Adelson, E, H. (2007).
    Image statistics and the perception of surface qualities.
    {\it Nature}, 447, 206-209.

\bibitem{Marlow1}
    Marlow, P, J., Todorovic, D., Anderson, B, L. (2015).
    Coupled computations of three-deimensional shape and material.
    {\it Current Biology}, 25(6):16, 221-222.

\bibitem{Marlow2}
    Marlow, P, J., Kim, J., Anderson, B, L. (2012).
    The Perception and Misperception of Specular Surface Reflectance.
    {\it Current Biology}, 22(20):23, 1909-1913.

\bibitem{HKeffect}
    Nakatani, Y. (1998).
    Simple estimation methods for the Helmholtz-Kohlrausch effect.
    {\it Color Research and Application}, 22(6), 385-401.

\bibitem{Nishida}
    Nishida, S., Motoyoshi, I., Nakano, L., Li, Y., Sharan, L., Adelson, E, H. (2008).
    Do colored highlights look like highlights?
    {\it Journal of Vision}, 8(6):339.

\bibitem{Hunter}
    Hunter, R, S. (1937).
    Methods of determining gloss.
    {\it Journal of Research of the National Bureau of Standards}, 18.

\bibitem{Psychtoolbox}
    Brainard, D. H. (1997).
    The psychophysics toolbox.
    {\it Spatial Vision}, 10, 433-436.

\bibitem{Mitsuba}
    Mitsuba - physically based renderer. https://www.mitsuba-renderer.org/

\bibitem{StanfordModels}
    Turk, G., Levoy, M. (2005).
    Stanford Computer Graphics Laboratory. https://graphics.stanford.edu/

\bibitem{Ward}
    Ward, G. J. (1992).
    Measuring and Modeling anisotropic reflection.
    {\it Proceedings of the 19th SIG-GRAPH}, 265-272.

\bibitem{uvYimage}
    Walker, J. Colour Rendering of Spectra. \\
    https://www.fourmilab.ch/documents/specrend/. (2020年1月30日閲覧)

\bibitem{LightMode}
    外山 寛之,大月 英雄,榊原 直彦,阿山 みよし(1997).
    光源色モードと物体色モードの色の見えの変化.
    {\it 映像情報メディア学会報告}, 21(28), 13-18.

\end{thebibliography}
