%\chapter{Conclusion}
\chapter{結論}

    本研究で得られた結論を以下に述べる.
    \begin{itemize}
        \item 輝度は同一であるが色度が異なるコンピュータグラフィックス画像間で光沢感を比較したところ,色度による有意な差が見られた.この結果から,光沢感知覚は輝度情報だけでは決定されず,色度情報も寄与していることが明らかになった.
        \item 拡散反射成分と鏡面反射成分の両方に色度を付与した刺激と拡散反射成分のみに色度を付与した刺激間で光沢感を比較したところ,色度による光沢感の傾向に大きな違いは見られなかった.もし鏡面反射成分と拡散反射成分の間の明るさコントラストが光沢感に寄与するのであれば,拡散反射成分のみに色度を付与した場合にはその明るさ感が増加することで鏡面/拡散反射成分間のコントラストが小さくなり,光沢感が減衰するはずであった.この結果から,光沢感に関して,鏡面反射成分と拡散反射成分の間の明るさ感コントラストの寄与は小さいことが示唆された.
        \item 刺激色度ごとに知覚される光沢感と明るさ感の関連性を調べたところ,それらに相関が見られたものの,黄色や紫色などにおいて明るさ感は小さいのに対して光沢感が顕著に高くなる傾向が見られた.この結果から,光沢感知覚に明るさ感も寄与している可能性が高いこと,ただし,刺激の色度も直接的に光沢感知覚に影響を与える可能性が示唆された.
    \end{itemize}
    \newpage