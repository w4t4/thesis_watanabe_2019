ヒトは物体を見ると,その光沢感を感じることができる.
従来研究より,様々な輝度統計量が光沢感の手がかりとなることが示唆されてきた.
例えば,輝度のサブバンド成分の歪度やコントラスト,また鏡面反射成分と拡散反射成分との輝度コントラストなどが光沢感と相関することはよく知られている.
一方で,色度と光沢感の関連性に関する研究はあまり多くはない.
鏡面反射成分に極端に強い彩度を持つ色度を付与した場合には,光沢感が急激に消失するという報告はあるが,これは一般的な視環境における色度情報の役割に関する研究ではない.
例えば,物体表面に色度が存在するとヘルムホルツ・コールラウシュ効果によって知覚的明るさが変化するため,もし光沢感に寄与するのが輝度成分そのものではなく知覚的明るさであるのなら,色度情報により光沢感が変化する可能性も十分に考えられる.

そこで本研究では,光沢感知覚に対する色度情報の寄与を心理物理実験により検討した.
実験刺激として,輝度は同一であるが色度のみが異なるコンピュータグラフィックス画像を用いた.
色度のみが異なる二刺激を対にして呈示し,どちらの刺激の光沢感が高いかを応答してもらうことにより,各刺激に対する光沢感を定量化した.
実験の結果から,色度により光沢感が有意に異なることが明らかとなった.
そこで,追加実験として各色度に対する明るさ感を計測し,光沢感との相関関係を調べたところ,相関はあるものの,色度による光沢感変化は明るさ感の変化だけでは説明できなかった.
これらの結果から,色度情報が光沢感に影響すること,ならびに,その影響の一部は明るさ感によるものであるが,それ以外に色度が直接的に光沢感に影響を与える可能性が示唆された.