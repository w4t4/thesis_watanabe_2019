ヒトは物体を見た時に, 光沢感を把握することができる.
従来研究より,鏡面反射成分とその他の成分との輝度コントラストが光沢感に寄与していることが示されている.
しかし,物体の色度によっては輝度と明るさ感が一致するとは限らない.
光沢感に輝度が寄与しているか明るさ感が寄与しているかは不明である.

そこで本研究では,光沢感知覚に対する色度情報の寄与を心理物理実験により検討した.
被験者に,輝度は同一であるが色度が異なる刺激を呈示し,光沢感を測定した.
この実験の結果から,刺激全体の明るさ感が光沢感に寄与している可能性が示されたため,刺激から知覚される光沢感と明るさ感の関連性を調べた.
その結果,明るさが光沢感に寄与している可能性が高いが,その他の要因も光沢感に寄与している可能性が示唆された.
