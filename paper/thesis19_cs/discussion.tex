\chapter{総合考察}
    \section{光沢感知覚に対する色度情報の寄与}
        本研究で行った実験から,色度情報が光沢感知覚に寄与することが示唆された.
        本研究の目的は,色度情報が光沢感に寄与するかどうか,また寄与する場合にはどのような要因が考えられるかを心理物理実験により明らかにすることであった.

        被験者にコンピュータグラフィックスとして生成された,輝度は同一であるが色度が異なる複数の画像刺激を呈示した.
        明るさ感のコントラストが光沢感に寄与しているという仮説をもとに,拡散反射成分と鏡面反射成分の両方に色度を付与するSD条件,拡散反射成分のみに色度を付与するD条件の2種類の条件を設定し,それらの条件をStanford DragonとStanford Bunnyの2種類の形状にそれぞれ適用した.
        被験者は,これらの刺激において,光沢感をより感じられる方を選択し,形状と条件間において色度ごとの光沢感を定量化した.

        この実験の結果,各形状と条件間で色度ごとの光沢感が異なることから,光沢感知覚に色度情報が寄与していることが示された.
        また,SD条件とD条件において応答の傾向に大きな違いが見られないことから,明るさ感のコントラストが光沢感に寄与しているという仮説が棄却された.
        加えて,刺激全体の明るさ感が光沢感に寄与している可能性が示唆された.

        次に,刺激全体の明るさ感が光沢感に寄与しているかを調べるために,上記の実験で用いた刺激の明るさ感を,刺激の平均色のパッチを用いることで測定した.
        さらに,測定された光沢感と明るさ感の関連性を調べることで,光沢感への明るさ感の寄与の度合いを算出した.

        この実験の結果では,Dragon形状のSD条件における明るさと光沢感の相関に強い正の相関があり,Bunny形状のD条件における明るさ感と光沢感の相関に比較的弱い正の相関があった.
        このことから,光沢感に明るさ感が寄与する可能性が高いが,それ以外の別の要因も光沢感に寄与している可能性が示唆された.


    \section{今後の課題}
        本研究では光沢感に明るさ感以外の要因が寄与している可能性が示唆されたが,具体的にそれがどのようなものであるかを明らかにすることはできなかった.


    \newpage