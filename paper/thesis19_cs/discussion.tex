\chapter{総合考察}
    \section{光沢感知覚に対する色度情報の寄与}
        本研究では,光沢感知覚が輝度情報のみによって決定されるのか,それとも色度情報も寄与するのかを明らかにすることを目的として,輝度は同一だが色度のみが異なるコンピュータグラフィックス画像を刺激として用いて心理物理実験を行った.
        [何をやったか]
        実験結果では,色度によって結に光沢感が異なっていた.
        この結果から,色度情報が光沢感知覚に寄与することが示唆された.

        本研究の大きな目的は,光沢感知覚が輝度情報のみによって決定されるのか,それとも色度情報も寄与するのかを心理物理実験により明らかにすることであった.
        また,色度情報が光沢感に寄与する場合には,どのようなメカニズムが関与しているかについても,特に知覚的明るさという観点から検討した.

        輝度は同一であるが色度が異なる複数のコンピュータグラフィックス画像を刺激として用い,サーストンの一対比較法により各刺激に対する光沢感を定量化した.
        実験の結果,図???や図???からわかるように,すべての物体形状とD条件(拡散反射成分のみに色度を付与する条件)/SD 条件(刺激全体に色度を付与する条件)において,色度により光沢感が有意に異なっていた.
        この結果から,光沢感知覚は輝度だけでは決まらず,色度情報が寄与していることが示された.
        [追記箇所: 先行研究との対応,新規性の主張]

    \section{光沢感知覚に対する色度情報の寄与}
        本実験では、従来研究におけるハイライトコントラストの重要性に立脚し、拡散反射成分に対する鏡面ハイライトの明るさコントラストが光沢感に寄与するという仮説についても合わせて検討した。
        この仮説が正しければ、D条件においては色度情報により拡散反射成分のみ明るさ感が高くなり、結果として光沢感が低くなるはずであった。
        しかし、実験結果では、SD条件とD条件のどちらにおいても、応答の傾向に大きな違いは見られず、色度情報を付与することで光沢感が向上した。
        この結果から、上記の仮説は棄却されることになる。

        D条件で光沢感が向上した理由は何だろうか。その1つの可能性として、鏡面ハイライトの明るさコントラストだけではなく、「見えのコントラスト」が効いたことが考えられる。
        D条件の刺激を見ると、たしかにハイライトの明るさコントラストは減少しているものの、一方で色度を付与したことにより色コントラストが追加され、結果として鏡面ハイライトと拡散反射の見かけ上のコントラストはむしろ増加しているように見える。
        [追記箇所]

    \section{色度情報の寄与と明るさ感の影響}
        色度情報はどのように光沢感に寄与したのだろうか。
        前節で述べたとおり、明るさコントラストを通じて光沢感に寄与したわけではないはずである。
        可能性の1つとしては、刺激から知覚される明るさ感の影響が考えられた。今回は背景が黒であったこともあり、刺激の明るさ感が高ければ、それがいわゆる発光色モード[参考文献]であるように知覚され、これが光沢感を増強する方向に働いたかも知れない。
        
        この可能性を検討するために、実験刺激で用いた色度による明るさ感増強効果を、均一色パッチ刺激を用いて測定した。
        さらに、測定された光沢感と明るさ感の関連性を調べることで、光沢感への明るさ感の寄与の度合いを算出した。この実験の結果では、Dragon形状のSD条件に置いて明るさと光沢感の相関に強い正の相関が・・・
        このことから、光沢感も明るさ感にある程度寄与していることが示唆される。

        しかし、明るさ感と光沢感の間に相関はあるものの、明確にそれらの関係性が崩れる色相があった。
        例えば、黄色や紫色などにおいて[色修正]、明るさ感は低いにも関わらず光沢感は高くなっていた。
        これらの色相における光沢感変化は明るさ感からは予測できないため、明るさ感以外の要因に起因すると考えられる。
        その要因の候補として、色度そのものが直接的に光沢感を変調させた可能性が考えられる。
        [追記箇所]

    \section{今後の課題}
        本研究では光沢感に明るさ感以外の要因が寄与している可能性が示唆されたが、具体的にそれがどのようなものであるかを同定するには至っていない。
        その候補として、[追記箇所]が考えられる。
        これらの影響を直接的に心理物理実験により明らかにしていく必要がある。

        また、応用を考える上では、色度と光沢感の関係性をより詳細にマッピングする必要がある。
        今回の実験では、ごく限られた色度条件による光沢感変化しか調べていない。
        しかし、色度情報で光沢感を操作することを考えれば、当然D条件とSD条件の光沢感の違いも定量化する必要がある。
        また従来の研究で知られていた高彩度色の付与に寄る光沢感喪失\cite{Nishida}を考慮すれば、刺激彩度の影響も明らかにする必要がある。
        これらを包括的に理解することにより、光沢知覚メカニズムについて詳細な検討ができるようになることに加え、画像情報からの光沢感の予測技術や、彩色による光沢感操作技術など、様々な応用研究が可能になると考えられる。


    \newpage