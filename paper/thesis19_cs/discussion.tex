\chapter{総合考察}
    \section{光沢感知覚に対する色度情報の寄与}
        本研究の大きな目的は,光沢感知覚が輝度情報のみによって決定されるのか,それとも色度情報も寄与するのかを心理物理実験により明らかにすることであった.
        また,色度情報が光沢感に寄与する場合には,どのようなメカニズムが関与しているかについても,特に知覚的明るさという観点から検討した.
        
        本研究で行った心理物理実験では,輝度は同一であるが色度が異なる複数のコンピュータグラフィックス画像を刺激として用い,サーストンの一対比較法により各刺激に対する光沢感を定量化した.
        実験の結果,図\ref{ex1_DSD}-図\ref{ex1_BD}からわかるように,すべての物体形状とD条件(拡散反射成分のみに色度を付与する条件)/SD条件(刺激全体に色度を付与する条件)において,色度により光沢感が有意に異なっていた.
        この結果から,光沢感知覚は輝度だけでは決まらず,色度情報が寄与していることが明らかになった.

    \section{ハイライトの明るさ感コントラストの影響}
        色度情報は,どのような機序により光沢感知覚に寄与したのだろうか.
        本研究では,従来研究におけるハイライトコントラストの重要性に立脚し,拡散反射成分に対する鏡面ハイライトの明るさ感コントラストが光沢感に寄与するという仮説についても合わせて検討した.
        この仮説が正しければ,D条件においては色度情報により拡散反射成分のみ明るさ感が高くなり,鏡面ハイライトの明るさ感コントラストが減少することで,光沢感が低くなると予想された.
        一方で,SD条件では,刺激全体の明るさが変化するのみでハイライトの明るさ感コントラストは変化しないため,光沢感への影響がないと予想された.
        しかし,実験結果では,SD条件とD条件のどちらにおいても,色度情報を付与することで光沢感が同程度向上した.
        この結果から,上記の仮説は棄却された.

        D条件で光沢感が向上した理由は何だろうか.
        その1つの可能性として,鏡面ハイライトの明るさ感コントラストだけではなく,「見えのコントラスト」が効いたことが考えられる.
        D条件の刺激を見ると,確かにハイライトの明るさ感コントラストは減少しているものの,一方で色度を付与したことにより色コントラストが追加され,結果として鏡面ハイライトと拡散反射の見かけ上のコントラストはむしろ増加しているように見える.
        したがって,鏡面ハイライトと拡散反射成分の総合的な見かけ上のコントラストが光沢感に寄与している可能性がある.
        この見かけ上のコントラストは,当然のように色度によって異なり,本研究の結果からは予想が困難である.
        この色度と見かけ上のコントラストの関係性を定量的に調べることは今後の課題である.

    \section{色度情報の寄与と明るさ感の影響}
        前節で述べたとおり,色度情報は明るさ感コントラストを通じて光沢感に寄与したわけではないと考えられる.
        可能性の1つとしては,刺激から知覚される明るさ感の影響が考えられた.今回は背景が黒であったこともあり,刺激の明るさ感が高ければ,それがいわゆる発光色モード\cite{LightMode}であるように知覚され,これが光沢感を増強する方向に働いたかもしれない.
        
        この可能性を検討するために,実験2では,実験1で用いた刺激の色度による明るさ感増強効果を,均一色パッチ刺激を用いて測定した.
        さらに,測定された光沢感と明るさ感の関連性を調べることで,光沢感への明るさ感の寄与の度合いを算出した.
        この実験の結果では,D条件とSD条件のいずれにおいても,明るさ感と光沢感の間に強い正の相関があった.
        このことから,色度を刺激に付与した場合,色度付与による明るさ感変化が光沢感にある程度寄与していることが示唆される.

        しかし,明るさ感と光沢感の間に相関はあるものの,明確にそれらの関係性が崩れる色相があった.
        例えば,yellowやcyanにおける明るさ感は低いにも関わらず光沢感は高く,blueにおける明るさ感は高いにも関わらず光沢感は低かった.
        この結果は,これらの色相においては光沢感変化は明るさ感からは予測できないことを示しており,これらの色相では明るさ感以外の要因が光沢感を変調したと考えられる.
        その要因の候補として,色度そのものが直接的に光沢感を変調させた可能性が考えられる.
        
        本研究で用いた物体はDragon,Bunnyの2種類であった.
        この形状に対して黄色に彩色を行った刺激が,金の質感を有する置物のように感じられた可能性がある.
        他の色度の刺激と比べると,光沢がある物体の色としてなじみがあり,このことが光沢感知覚を増幅させた可能性が考えられる.
        
        また,Bunny形状とDragon形状では光沢感と明るさ感の相関の強さが異なり,Dragon形状においてその相関係数が大きかった.
        これはDragon形状とBunny形状の画像特徴の差異が,明るさ感による光沢感増幅効果の度合いに影響を与えた可能性が考えられる.
        今後,これらの画像の輝度ヒストグラムや空間周波数成分などの特徴を見ることで,色度が光沢感に有効に働くための条件が見いだせるかもしれない.

    \section{今後の課題}
        本研究では光沢感に明るさ感以外の要因が寄与している可能性が示唆されたが,具体的にそれがどのようなものであるかを同定するには至っていない.
        その候補として,自然シーンにおける鏡面反射成分と色度の関係性や,それらのヒトにとっての調和性の影響が考えられる.
        これらの影響を直接的に心理物理実験により明らかにしていく必要がある.

        また,応用を考える上では,色度と光沢感の関係性をより詳細にマッピングする必要がある.
        今回の実験では,ごく限られた色度条件による光沢感変化しか調べていない.
        しかし,色度情報で光沢感を操作することを考えれば,当然D条件とSD条件の光沢感の違いも定量化する必要がある.
        また従来の研究で知られていた高彩度色の付与による光沢感喪失\cite{Nishida}を考慮すれば,刺激彩度の影響も明らかにする必要がある.
        これらを包括的に理解することにより,光沢知覚メカニズムについて詳細な検討ができるようになることに加え,画像情報からの光沢感の予測技術や,彩色による光沢感操作技術など,様々な応用研究が可能になると考えられる.

    \newpage