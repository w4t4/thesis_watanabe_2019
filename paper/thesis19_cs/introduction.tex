%\chapter{Introduction}
\chapter{序論}

    ヒトは物体を見た時に, 瞬時にその質感を把握することができる。
    網膜像を生み出す物理的要因として、照明環境、物体の光学特性、物体の三次元形状があるが、それらは強くかつ複雑に相互作用する。
    この中で、質感とは主に物体の光学特性に対応する知覚であると考えることができる。
    この枠組みで考えれば、網膜像から質感を推定する問題はいわゆる不良設定問題であるにも関わらず、ヒトが容易に質感を知覚できることから、その仕組みを明らかにするための研究がここ15年ほど活発に行われてきた。

    質感の種類は非常に多種多様であるが[todo: 引用]、その質感の中でも、光沢感を知覚する際の視覚系の情報処理の仕組みについては多く研究されてきた。
    光沢感とは、物体の表面の光学的反射特性に対応する心理的な属性である。
    例えば、パチンコ玉のように表面が非常に平滑な平面においては、鏡面反射が支配的であり、周辺環境の明瞭な像が映り込むことにより非常に強い光沢感が知覚される。
    一方で布のように表面に凹凸が多く存在する平面においては、拡散反射が支配的であり、光沢感はほとんど知覚されない。

    光沢感については、多くの場合その輝度条件と光沢感知覚の関連性について多くの報告がなされてきた。[todo: 先行研究を引用]

    例えば, 物体における鏡面反射成分の面積・コントラスト・シャープネスが光沢知覚に寄与することが先行研究によって明らかとなっている.


    
    しかし、輝度と知覚的な明るさは必ずしも一致するとは限らない。
    例えば、刺激の輝度は同一でも、Helmholtz-Kohlrausch効果による明るさの増幅は色相によって異なることが知られている。
    では、光沢知覚には輝度と明るさ感のうちどちらが寄与しているのだろうか。
    また、Nishidaら(2010)はハイライトの色をつけすぎるとハイライトがハイライトに見えなくなり、それに伴い光沢感が大幅に減衰することを報告した。
    しかし、それは極端に不自然な例であり、日常の光沢感知覚に対する色情報の寄与に関する研究ではない。